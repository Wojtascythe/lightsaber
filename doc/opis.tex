% !TeX encoding = UTF-8
% !TeX spellcheck = pl_PL

% $Id:$

%Author: Wojciech Domski
%Szablon do ząłożeń projektowych, raportu i dokumentacji z sterowników robotów
%Wersja v.1.0.0
%


%% Konfiguracja:
\newcommand{\kurs}{Sterowniki robot\'{o}w}
\newcommand{\formakursu}{Projekt}

%odkomentuj właściwy typ projektu
\newcommand{\doctype}{Za\l{}o\.{z}enia projektowe}
%\newcommand{\doctype}{Raport}
%\newcommand{\doctype}{Dokumentacja}

%wpisz nazwę projektu
\newcommand{\projectname}{Miecz \'{S}wietlny}

%wpisz akronim projektu
\newcommand{\acronim}{Mi\'{S}}

%zmaiast X wpisz numer grupy projektowej
\newcommand{\nrgrupy}{4}
%wpisz Imię i nazwisko oraz numer albumu
\newcommand{\osobaA}{Patryk \textsc{Knapik}, 226302}
%w przypadku projektu jednoosobowego usuń zawartość nowej komendy
\newcommand{\osobaB}{Wojciech \textsc{Kosicki}, 234506}

%wpisz termin w formie, jak poniżej dzień, parzystość, godzina
\newcommand{\termin}{wtTP11}

%wpisz imię i nazwisko prowadzącego
\newcommand{\prowadzacy}{mgr in\.{z}. Wojciech \textsc{Domski}}

\documentclass[10pt, a4paper]{article}
% W nawiasie klamrowym podana jest klasa dokumentu. Standardowe klasy artykułu
% to: article, amsart, scrartcl, artikel1, artikel2, artikel3.
% W nawiasie prostokątnym deklarowane są opcje dokumentu. Zamiast 10pt
% można podać 11pt lub 12pt. Dokument w dwóch kolumnach uzyskuje się po
% wpisaniu opcji twocolumn, 

\include{preambula}
	
\begin{document}

\def\tablename{Tabela}	%zmienienie nazwy tabel z Tablica na Tabela

\begin{titlepage}
	\begin{center}
		\textsc{\LARGE \formakursu}\\[1cm]		
		\textsc{\Large \kurs}\\[0.5cm]		
		\rule{\textwidth}{0.08cm}\\[0.4cm]
		{\huge \bfseries \doctype}\\[1cm]
		{\huge \bfseries \projectname}\\[0.5cm]
		{\huge \bfseries \acronim}\\[0.4cm]
		\rule{\textwidth}{0.08cm}\\[1cm]
		
		\begin{flushright} \large
		\emph{Skład grupy (\nrgrupy):}\\
		\osobaA\\
		\osobaB\\[0.4cm]
		
		\emph{Termin: }\termin\\[0.4cm]

		\emph{Prowadzący:} \\
		\prowadzacy \\
		
		\end{flushright}
		
		\vfill
		
		{\large \today}
	\end{center}	
\end{titlepage}

\newpage
\tableofcontents
\newpage

\section{Opis projektu}
\label{sec:OpisProjektu}
Projekt zakłada stworzenie urządzenia które będzie wykorzystywało dane z akcelerometru do generowania dźwięków o odpowiedniej częstotliwości za pomocą układu DAC. Dźwięki będą generowane w taki sposób, żeby imitowały pracę poruszanego miecza świetlnego znanego chociażby z filmów pt. \textit{Star Wars}. 

\section{Założenia projektowe} %jak będzie buczeć inaczej w idlu i inaczej podczas ruszania to już będę zadowolony
	\begin{flushleft}
	Założenia podstawowe funkcjonalności:\\
	\end{flushleft}
	\begin{itemize}
		\item Dioda LED RGB imitująca ostrze miecza.
		\item Wybór koloru diody przez użytkownika za pomocą przycisku z palety kolorów w jakich najczęściej występują miecze świetlne.
		\item Miecz ma generować dźwięk, przypominający ten znany z miecza świetlnego, za pomocą DACa - ruch mieczem ma zmieniać ton i częstotliwość dzwięku w podobny sposób jak robi to miecz świetlny.
		\item Możliwość włączania i wyłączania miecza za pomocą przycisku.
		\item Miecz wyłączony nie generuje dźwięków, a także wyłącza LED RGB.
	\end{itemize}
	\vspace{5mm}
	\begin{flushleft}
	Założenia dodatkowe funkcjonalności (wdrażane po tym jak założenia podstawowe funkcjonalności zostaną spełnione):\\
	\end{flushleft}
	\begin{itemize}
		\item Możliwość wyboru koloru z palety True color (24-bit).
		\item Dźwięk włączania i wyłączania miecza odtwarzany z zewnętrznej pamięci flash komunikującej się z mikrokontrolerem za pomocą QSPI.
		\item Szkielet na baterię (np. power bank USB), głośnik oraz pasek LED RGB.
	\end{itemize}
	\vspace{5mm}
	\begin{flushleft}
	Założenia budowy programu:\\
	\end{flushleft}
	\begin{itemize}
		\item Implementacja bazująca na przerwaniach - główna pętla obsługuje zdarzenia po ustawieniu flagi w przerwaniu.
		\item Wykorzystanie programu \textit{CubeMX} dostarczanego przez firmę \textit{ST} w celu wygenerowania kodu potrzebnego do w/w peryferiów.
	\end{itemize}

	

\section{Harmonogram pracy} %tutaj powinny być chyba kamienie milowe i wykres Gantta, DODAC TERMINY!!
	\begin{flushleft}
	Kamienie milowe zostały pogrubione i podkreślone\\
	\end{flushleft}
	\begin{enumerate}
		\item Wygenerowanie projektu bazowego w CubeMX.\\
		\underline {Termin: 06.03 - 08.03}
		\item Implementacja obsługi DACa.\\
		\underline {Termin: 09.03 - 22.03}
		\item Implementacja obsługi akcelerometru.\\
		\underline {Termin: 09.03 - 22.03}
		\item Opracowanie algorytmu generującego dźwięk zbliżony do dźwięku miecza świetlnego.\\
		\underline {Termin: 15.03 - 29.03}
		\item Obsługa diody LED RGB imitującej ostrze za pomocą PWM.\\
		\underline {Termin: 30.03 - 02.04}
		\textbf{\item \underline{Implementacje głównej pętli programu (połączenie akcelerometru z układem DAC).}}\\
		\underline {Termin: 03.04 - 09.04}
		\item Opracowanie wstępnej dokumentacji technicznej po spełnieniu założeń podstawowych.\\
		\underline {Termin: 10.04 - 17.04}
		\item Implementacja obsługi pamięci flash oraz wgrywanie dźwięków za pomocą USB.\\
		\underline {Termin: 18.04 - 02.05}
		\item Implementacja odgrywania dźwięków włączania i wyłączania miecza świetlnego.\\
		\underline {Termin: 03.05 - 10.05}
		\item Implementacja wyboru koloru z palety True color (24-bit) - obsługa przycisków, wyświetlacza i PWM.\\
		\underline {Termin: 18.04 - 25.04}
		\textbf{\item \underline{Rozbudowa głównej pętli programu o obsługę pamięci flash i wyboru koloru z palety.}}\\
		\underline {Termin: 11.05 - 18.05}
		\item Aktualizacja dokumentacji technicznej po spełnieniu założeń dodatkowych.\\
		\underline {Termin: 26.04 - 22.05}
		\item Budowa ramy na komponenty oraz ostrze.\\
		\underline {Termin: 23.05 - 31.05}
	\end{enumerate}
	\begin{figure}[h]
	\centering
	\includegraphics[width=\linewidth]{gantt.png}
	\caption{Wykres Gantta}
	\end{figure}
	
	
	\section{Opis kamieni milowych}
	
	\begin{enumerate}
	
	\item \textbf{Implementacje głównej pętli programu (połączenie akcelerometru z układem DAC).}
	Duża część pracy związana z tym punktem, będzie zawierać się w podstawowej implementacji przyrządów zewnętrznych. Gdy etap implementacji obsługi akcelerometru i DACa zostanie zakończony, to kolejnym krokiem będzie implementacja podstawowej pętli, na bazie której układ będzie generował dźwięk w zależności od ruchu.
	
	\item \textbf{Rozbudowa głównej pętli programu o obsługę pamięci flash i wyboru koloru z palety.}
	Pierwszy kamień milowy jest podstawą do działania całego projektu. W rozszerzonej wersji planowane jest odtwarzanie dźwięków których generowanie za pomocą mikrokontrolera jest bardzo trudne, mianowicie dźwięki otwierania, zamykania i uderzenia miecza. Dodatkową funkcjonalnością jest możliwość wyboru dowolnego koloru z palety True color (24-bit) przez użytkownika. Kiedy ten etap zostanie zrealizowany będzie można zbudować prototypową ramę na ostrze oraz urządzenia niezbędne do samowystarczalnego działania miecza.
	
	\end{enumerate}
	
\newpage
\section{Podział pracy} %tutaj też zakres prac 
	
	\begin{itemize}
		\item Patryk Knapik:
			\begin{enumerate}
				\item Implementacja obsługi DAC.
				\item Implementacja obsługi pamięci flash oraz komunikacji USB.
			\end{enumerate}
		\item Wojciech Kosicki:
			\begin{enumerate}
				\item Implementacja obsługi akcelerometru.
				\item Implementacja wyboru koloru z palety True color (24-bit).
				\item Budowa ramy.
			\end{enumerate}
		\item Zadania wspólne:
			\begin{enumerate}
				\item Opracowanie algorytmu generującego dźwięk zbliżony do dźwięku miecza świetlnego.
				\item Implementacja głównej pętli programu.
				\item Opracowanie wstępnej dokumentacji technicznej.
				\item Rozbudowa głównej pętli programu.
				\item Opracowanie finalnej dokumentacji technicznej.
			\end{enumerate}
	\end{itemize}

\section{Podsumowanie}
Temat projektu został dobrany w taki sposób, aby opanować podstawowe umiejętności programowania mikrokontrolerów STM32 jednocześnie tworząc coś ciekawego i zjawiskowego. Na całym świecie miliony fanów szermierki na miecze świetlne walczą ze sobą bronią symulującą wygląd miecza świetlnego. Pomimo takiej popularności i widowiskowości, nie stworzono do tej pory systemu, który by generował charakterystyczny dźwięk tego oręża. Projekt będzie próbą wyjścia na przeciw tym problemom. 

Z uwagi na mnogość peryferiów i małe doświadczenie konstruktorów, projekt ten może być nie małym wyzwaniem.
Niezależnie od komplikacji, plan realizacji zadań projektowych został podzielony pomiędzy dwójkę wykonującą zadanie, w zamyśle był równy podział prac, więc może podlegać on zmianom w taki sposób, żeby to założenie zostało spełnione.
\newpage
\addcontentsline{toc}{section}{Bibilografia}
\bibliography{bibliografia}
\bibliographystyle{plain}


\end{document}







































